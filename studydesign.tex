% Options for packages loaded elsewhere
\PassOptionsToPackage{unicode}{hyperref}
\PassOptionsToPackage{hyphens}{url}
\PassOptionsToPackage{dvipsnames,svgnames,x11names}{xcolor}
%
\documentclass[
  letterpaper,
  DIV=11,
  numbers=noendperiod]{scrartcl}

\usepackage{amsmath,amssymb}
\usepackage{iftex}
\ifPDFTeX
  \usepackage[T1]{fontenc}
  \usepackage[utf8]{inputenc}
  \usepackage{textcomp} % provide euro and other symbols
\else % if luatex or xetex
  \usepackage{unicode-math}
  \defaultfontfeatures{Scale=MatchLowercase}
  \defaultfontfeatures[\rmfamily]{Ligatures=TeX,Scale=1}
\fi
\usepackage{lmodern}
\ifPDFTeX\else  
    % xetex/luatex font selection
\fi
% Use upquote if available, for straight quotes in verbatim environments
\IfFileExists{upquote.sty}{\usepackage{upquote}}{}
\IfFileExists{microtype.sty}{% use microtype if available
  \usepackage[]{microtype}
  \UseMicrotypeSet[protrusion]{basicmath} % disable protrusion for tt fonts
}{}
\makeatletter
\@ifundefined{KOMAClassName}{% if non-KOMA class
  \IfFileExists{parskip.sty}{%
    \usepackage{parskip}
  }{% else
    \setlength{\parindent}{0pt}
    \setlength{\parskip}{6pt plus 2pt minus 1pt}}
}{% if KOMA class
  \KOMAoptions{parskip=half}}
\makeatother
\usepackage{xcolor}
\setlength{\emergencystretch}{3em} % prevent overfull lines
\setcounter{secnumdepth}{-\maxdimen} % remove section numbering
% Make \paragraph and \subparagraph free-standing
\ifx\paragraph\undefined\else
  \let\oldparagraph\paragraph
  \renewcommand{\paragraph}[1]{\oldparagraph{#1}\mbox{}}
\fi
\ifx\subparagraph\undefined\else
  \let\oldsubparagraph\subparagraph
  \renewcommand{\subparagraph}[1]{\oldsubparagraph{#1}\mbox{}}
\fi


\providecommand{\tightlist}{%
  \setlength{\itemsep}{0pt}\setlength{\parskip}{0pt}}\usepackage{longtable,booktabs,array}
\usepackage{calc} % for calculating minipage widths
% Correct order of tables after \paragraph or \subparagraph
\usepackage{etoolbox}
\makeatletter
\patchcmd\longtable{\par}{\if@noskipsec\mbox{}\fi\par}{}{}
\makeatother
% Allow footnotes in longtable head/foot
\IfFileExists{footnotehyper.sty}{\usepackage{footnotehyper}}{\usepackage{footnote}}
\makesavenoteenv{longtable}
\usepackage{graphicx}
\makeatletter
\def\maxwidth{\ifdim\Gin@nat@width>\linewidth\linewidth\else\Gin@nat@width\fi}
\def\maxheight{\ifdim\Gin@nat@height>\textheight\textheight\else\Gin@nat@height\fi}
\makeatother
% Scale images if necessary, so that they will not overflow the page
% margins by default, and it is still possible to overwrite the defaults
% using explicit options in \includegraphics[width, height, ...]{}
\setkeys{Gin}{width=\maxwidth,height=\maxheight,keepaspectratio}
% Set default figure placement to htbp
\makeatletter
\def\fps@figure{htbp}
\makeatother
\newlength{\cslhangindent}
\setlength{\cslhangindent}{1.5em}
\newlength{\csllabelwidth}
\setlength{\csllabelwidth}{3em}
\newlength{\cslentryspacingunit} % times entry-spacing
\setlength{\cslentryspacingunit}{\parskip}
\newenvironment{CSLReferences}[2] % #1 hanging-ident, #2 entry spacing
 {% don't indent paragraphs
  \setlength{\parindent}{0pt}
  % turn on hanging indent if param 1 is 1
  \ifodd #1
  \let\oldpar\par
  \def\par{\hangindent=\cslhangindent\oldpar}
  \fi
  % set entry spacing
  \setlength{\parskip}{#2\cslentryspacingunit}
 }%
 {}
\usepackage{calc}
\newcommand{\CSLBlock}[1]{#1\hfill\break}
\newcommand{\CSLLeftMargin}[1]{\parbox[t]{\csllabelwidth}{#1}}
\newcommand{\CSLRightInline}[1]{\parbox[t]{\linewidth - \csllabelwidth}{#1}\break}
\newcommand{\CSLIndent}[1]{\hspace{\cslhangindent}#1}

\KOMAoption{captions}{tableheading}
\makeatletter
\makeatother
\makeatletter
\makeatother
\makeatletter
\@ifpackageloaded{caption}{}{\usepackage{caption}}
\AtBeginDocument{%
\ifdefined\contentsname
  \renewcommand*\contentsname{Table of contents}
\else
  \newcommand\contentsname{Table of contents}
\fi
\ifdefined\listfigurename
  \renewcommand*\listfigurename{List of Figures}
\else
  \newcommand\listfigurename{List of Figures}
\fi
\ifdefined\listtablename
  \renewcommand*\listtablename{List of Tables}
\else
  \newcommand\listtablename{List of Tables}
\fi
\ifdefined\figurename
  \renewcommand*\figurename{Figure}
\else
  \newcommand\figurename{Figure}
\fi
\ifdefined\tablename
  \renewcommand*\tablename{Table}
\else
  \newcommand\tablename{Table}
\fi
}
\@ifpackageloaded{float}{}{\usepackage{float}}
\floatstyle{ruled}
\@ifundefined{c@chapter}{\newfloat{codelisting}{h}{lop}}{\newfloat{codelisting}{h}{lop}[chapter]}
\floatname{codelisting}{Listing}
\newcommand*\listoflistings{\listof{codelisting}{List of Listings}}
\makeatother
\makeatletter
\@ifpackageloaded{caption}{}{\usepackage{caption}}
\@ifpackageloaded{subcaption}{}{\usepackage{subcaption}}
\makeatother
\makeatletter
\@ifpackageloaded{tcolorbox}{}{\usepackage[skins,breakable]{tcolorbox}}
\makeatother
\makeatletter
\@ifundefined{shadecolor}{\definecolor{shadecolor}{rgb}{.97, .97, .97}}
\makeatother
\makeatletter
\makeatother
\makeatletter
\makeatother
\ifLuaTeX
  \usepackage{selnolig}  % disable illegal ligatures
\fi
\IfFileExists{bookmark.sty}{\usepackage{bookmark}}{\usepackage{hyperref}}
\IfFileExists{xurl.sty}{\usepackage{xurl}}{} % add URL line breaks if available
\urlstyle{same} % disable monospaced font for URLs
\hypersetup{
  pdftitle={Study design},
  colorlinks=true,
  linkcolor={blue},
  filecolor={Maroon},
  citecolor={Blue},
  urlcolor={Blue},
  pdfcreator={LaTeX via pandoc}}

\title{Study design}
\author{}
\date{}

\begin{document}
\maketitle
\ifdefined\Shaded\renewenvironment{Shaded}{\begin{tcolorbox}[enhanced, sharp corners, boxrule=0pt, breakable, frame hidden, borderline west={3pt}{0pt}{shadecolor}, interior hidden]}{\end{tcolorbox}}\fi

Jeg har valgt å ta for meg et tema som omhandler intervalltrening og det
maksimale oksygenopptaket (VO\textsubscript{2maks}).
VO\textsubscript{2maks} kan defineres som kroppens maksimale evne til å
ta opp og omsette oksygen, altså den maksimale hastigheten på den aerobe
energiomsetningen. Flick's likning illustrerer de bestemmede faktorene
for VO\textsubscript{2maks} som er minuttvolum (MV), arteriell
O\textsubscript{2} (aO\textsubscript{2}) og venøs O\textsubscript{2}
(vO\textsubscript{2}).

VO2maks er assosiert med både helse og prestasjon, og det er derfor et
interessant tema som i større eller mindre grad angår alle. Kunnskap
knyttet til trening for å forbedre VO\textsubscript{2maks} er relevant
både når det kommer til mange former for toppidrett og prestasjon, men
også for generell helse blandt befolknigen. Intervalltrening er i mange
tilfeller en mer tidseffektiv treningsform som for mange kan være
lettere å sette av tid til i en hektisk hverdag. Dersom intervalltrening
har like gode eller bedre effekt på helsen enn en mer langvarig form for
trening, vil det være positivt for folkehelsen da flere har tid til det.
Samtidig har vi også en mindre del av beofolkningen som er opptatt av
prestasjon, og i deres tilfelle er det nyttig å vite hvilken
treningsform som belaster kroppen minst og krever minst restitusjon
samtidig som man får størst treningseffekt. I en toppidrettshverdag
handler det om å optimalisere alle faktorer i hverdagen for best mulig
prestasjon og restitusjon. Da er denne formen for kunnskap nyttig da den
kan bidra til å strukturere hverdagen og treningen på best mulig måte
slik at hver enkelt på best mulig måte kan legge opp treningen etter
sine egne behov.

Felles for alle studiene jeg har valgt ut er at de ønsker å undersøke
hvilken effekt intervallpreget trening har på VO\textsubscript{2maks}
sammenliknet med kontinuerlig arbeid. Studiene har litt ulike
tilnærminger til temaet med noe varierende hypoteser eller
problemstillinger. Daussin et al. (2007) undersøkte hypotesen om at
kontinuerlig trening og intervalltrening påvirker hhv. perifere og/eller
sentrale faktorer. Astorino et al. (2013) og Helgerud et al. (2007)
ønsket å sammenlikne endringer i det maksimale oksygenopptaket som
respons på ulike treningsregimer. Nybo et al. (2010) hadde fokus på
effektivitet i treningen og hadde som hensikt å undersøke effekten av
høyintensiv intervalltrening sammenliknet med kontinuerlig arbeid og
styrketrening på en rekke helseparametere, deriblandt det maksimale
oksygenopptaket. Matsuo et al. (2014) hadde på lik linje med Nybo fokus
på effektivitet og hadde som hensikt å der han sammenliknet tidseffektiv
høyintensiv intervalltrening og kontinuerlig arbeid. I tillegg til dette
er samtlige studier, med unntak av Helgerud et al.~gjort på utrente.
Studiene initierer at de tror intervalltrening utført med høy intensitet
vil gi bedre adaptasjoner enn kontinuerlig trening, men de ønsker å
undersøke hvor stor den eventuelle forskjellen er samt diskutere
årsaksforklaringer.

Flere at studiene har mange likhetstrekk i valg av metode, men det er
noen ulikheter. Alle studiene er intervensjonsstudier over en periode på
8-12 (28) uker med pre- og post-tester. Daussin et al. (2007) har også
en detreningsperiode på etterfulgt av en ny intervensjon der gruppene
bytter treningsform, og er den eneste av disse fem studiene som har en
randomisert crossover design. Astorino et al. (2013) og Nybo et al.
(2010) har begge en randomisert kontrollert between group design.
Helgerud et al. (2007) og Matsuo et al. (2014) hadde en randomisert
between group design, altså uten en kontrollgruppe. Det er verdt å nevne
at i stuien til Matsuo et al. (2014) er gruppene kun delvis randomisert
da gruppene er basert på deltakernes alder og VO2maks pre-intervensjon.
Hensikten med dette er å øke studiens pålitelighet ved å gjøre gruppene
så like som mulig for å sikre at eventuelle effekter av en intervensjon
ikke kan tilskrives forskjeller mellom gruppene.

I en crossover studie trenger man ikke like mange deltakere som i de
andre studiedesignene fordi hver deltaker gjennomfører begge
intervensjonene samtidig som de fungerer som sin egen kontroll. Ulemper
ved denne studiedesignen er at den varer over dobbelt så lenge som mange
andre intervensjonsstudier, som kan påvirke rekruttering av deltakere og
deltakernes motivasjon underveis. Det er også tenkelig at deltakerne kan
ha en ``carryover'' effekt av de førsste intervensjonen som de tar med
seg inn i den andre til tross for en detreningsperiode Hulley (2013). I
studien til Daussin et al. (2007) er denne perioden satt til 12 uker og
man tenker at deltakerne vil komme tilbake til baseline i løpet av denne
perioden slik at de begynner med den andre intevrensjonen med ``blanke
ark''. I de andre studiene trengs det et større utvalg fra populasjonen
som videre deles inn i to eller flere grupper der hver gruppe
gjennomfører ulik trening. Noen studier har også grupper som er
kontrollgrupper som ikke gjennomfører noen for for trening, men som
gjennomfører testene. På grunn av måten disse studiene utføres trengs
det flere deltakere, hvilket kan være utfordrende å få tak i. Fordelene
med en kontrollgruppe er at man kan kontrollere for at det ikke er andre
faktorer som kan ha påvirket resultatet. To av de radomiserte between
group design studiene har en kontrollgruppe og to har ikke det.

Det er noe varierende utvalg i studiene. Daussin et al. (2007) har et
utvalg på 10 (5 kvinner og 5 menn) i sin crossover studie med en
gjennomsnittsalder på 47 år, mens de resterende fire har et utvalg som
varierer fra 20-42 med gjennomsnittsaldere sm varierer fra 23-31 år. Det
er kun studien til Matsuo et al. (2014) som nevner hvordan deltakerne
ble rekruttert, resten av studiene nevner ikke dette. I denne studien
ble deltakerne rekrutert via lokal markedsføring på universitetet og i
lokalavisa. Alle studiene foruten studien til Helgerud et al. (2007) som
har moderat trente personer, har friske men utrente deltakere. Det er
også noe vaiasjon i kjønnsfordelingen. Daussin et al. (2007) hadde både
kvinnelige og mannlige deltakere (5 + 5), Nybo et al. (2010), Matsuo et
al. (2014) og Helgerud et al. (2007) hadde kunn menn, mens Astorino et
al. (2013) kun hadde kvinnelige deltakere. Det er en kjent sak at det
gjøres mer forskning på menn enn på kvinner, og det er derfor fint at
noen studier også kan representere kvinner. Hvorvidt utvalget
representerer den generelle befolkningen kan diskuteres, men utvalget
representerer i det minste den gruppen av befolkningen med de
deltakerkarakteristikkene utvalget har.

Hver av studiene har valgt seg ut en rekke tester de utførte både pre og
post-intervensjonen. Felles for alle er at de gjennomflører
VO\textsubscript{2maks}-tester både pre og post intervensjon. I tillegg
til dette har flere av studiene en rekke andre tester som f.eks.
laktatterskel, maks watt, maksimal arteriovenøs differanse og
kardiovaskulære tester som blodtrykk og hjertefrekvens i hvile. Astorino
et al. (2013) er den eneste av disse studiene som gjennomførte tester
underveis i intervensjonen. Dette ble gjort hver tredje uke for å regne
ut hvilken intensitet deltakerne skulle trene på.

Til slutt samler studiene resultatene etter intervensjonen er
gjennomført. Astorino et al. (2013) brukte en two way analysis of
variance with repeated emasures for å undersøke forskjeller i variablene
som respons på trening. De resterende studiene har brukt en two way
ANOVA with repeated measures for å undersøke forskjellene mellom
gruppene og resultatene deres. Matsuo et al. (2014) har også gjort en
power test for å undesøke hvor stor sample size de må ha for å øke den
statistiske styrken i studien sin ved at sannsynligheten for at de
resultatene de får er overførbare til dresten av befolkningen. De brukte
en priori power test og kom frem til at de måtte ha minst 11 i hver
gruppe. De rekrutterte derfor litt flere enn dette da de tok høyde for
at noen muligens ville droppe ut underveis. Astorino et al. (2013)
brukte en one way ANOVA til å undersøke forskjeller i
deltakerkarakteristikker ved baseline mellom gruppene for å ha tatt
høyde for dette når resultatene skal analyseres.

Alle studiene har kommet fram til at trening med høyere intensitet er
bedre for å øke VO\textsubscript{2maks} enn trening med lavere
intensitet. Gruppene som ha trent med lavere intensitet har i de fleste
tilfellene også økt sitt maksimale oksygenopptak, men gruppene med
høyere intensitet har økt signifikant mer. Det betyr ikke nødvendigvis
at den høyeste intensiteten er best. I studien til Matsuo et al. (2014)
fikk gruppen som trente \textless90\% VO\textsubscript{2maks} (HIAT) en
større forbedring enn gruppen som trente på 120\%
VO\textsubscript{2maks} (SIT). Arbeidstiden til HIAT gruppen var 13 min
mens den kun var 5 min, hvilket kan ha spilt en rolle i resultatet. Det
er nok lettere å opprettholde intensiteten HIAT-gruppen hadde over
lengre tid, som er essensielt for å oppnå de adaptasjonene som er
assosiert med et økt maksimalt oksygenopptak. De resterende studiene er
enige i at intervalltrening med høy intensitet er å foretrekke fremfor
både kontinuerlig trening og intervaller med lavere intensitet for å
forbedre VO\textsubscript{2maks} både hos friske utrente menn og
kvinner, og hos moderat trente menn som har de samme fysike trekkene som
gruppene som har utført nettopp disse studiene.

\hypertarget{refs}{}
\begin{CSLReferences}{1}{0}
\leavevmode\vadjust pre{\hypertarget{ref-astorino2013}{}}%
Astorino, Todd A., Matthew M. Schubert, Elyse Palumbo, Douglas Stirling,
David W. McMillan, Christina Cooper, Jackie Godinez, Donovan Martinez,
and Rachael Gallant. 2013. {``Magnitude and Time Course of Changes in
Maximal Oxygen Uptake in Response to Distinct Regimens of Chronic
Interval Training in Sedentary Women.''} \emph{European Journal of
Applied Physiology} 113 (9): 2361--69.
\url{https://doi.org/10.1007/s00421-013-2672-1}.

\leavevmode\vadjust pre{\hypertarget{ref-daussin2007}{}}%
Daussin, Frédéric N., Elodie Ponsot, Stéphane P. Dufour, Evelyne
Lonsdorfer-Wolf, Stéphane Doutreleau, Bernard Geny, François Piquard,
and Ruddy Richard. 2007. {``Improvement of
{\$}{\$}\dot{\textbraceleft}V{\textbraceright}\hbox{\textbraceleft}O{\textbraceright}{\_}{\textbraceleft}2
\max{\textbraceright},{\$}{\$}by Cardiac Output and Oxygen Extraction
Adaptation During Intermittent Versus Continuous Endurance Training.''}
\emph{European Journal of Applied Physiology} 101 (3): 377--83.
\url{https://doi.org/10.1007/s00421-007-0499-3}.

\leavevmode\vadjust pre{\hypertarget{ref-helgerud2007}{}}%
Helgerud, Jan, Kjetill Høydal, Eivind Wang, Trine Karlsen, Pålr Berg,
Marius Bjerkaas, Thomas Simonsen, et al. 2007. {``Aerobic High-Intensity
Intervals Improve v{\textperiodcentered}O2max More Than Moderate
Training.''} \emph{Medicine \& Science in Sports \& Exercise} 39 (4):
665. \url{https://doi.org/10.1249/mss.0b013e3180304570}.

\leavevmode\vadjust pre{\hypertarget{ref-designin2013}{}}%
Hulley, Stephen B., ed. 2013. \emph{Designing Clinical Research}. 4th
ed. Philadelphia: Wolters Kluwer/Lippincott Williams \& Wilkins.

\leavevmode\vadjust pre{\hypertarget{ref-matsuo2014}{}}%
Matsuo, Tomoaki, Kousaku Saotome, Satoshi Seino, Nobutake Shimojo, Akira
Matsushita, Motoyuki Iemitsu, Hiroshi Ohshima, Kiyoji Tanaka, and Chiaki
Mukai. 2014. {``Effects of a Low-Volume Aerobic-Type Interval Exercise
on v{\textperiodcentered}O2max and Cardiac Mass.''} \emph{Medicine \&
Science in Sports \& Exercise} 46 (1): 42.
\url{https://doi.org/10.1249/MSS.0b013e3182a38da8}.

\leavevmode\vadjust pre{\hypertarget{ref-nybo2010}{}}%
Nybo, Lars, Emil Sundstrup, Markus D. Jakobsen, Magni Mohr, Therese
Hornstrup, Lene Simonsen, Jens Bülow, et al. 2010. {``High-Intensity
Training Versus Traditional Exercise Interventions for Promoting
Health.''} \emph{Medicine \& Science in Sports \& Exercise} 42 (10):
1951. \url{https://doi.org/10.1249/MSS.0b013e3181d99203}.

\end{CSLReferences}



\end{document}
